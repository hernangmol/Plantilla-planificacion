\documentclass[
11pt, % The default document font size, options: 10pt, 11pt, 12pt
codirector, % Uncomment to add a codirector to the title page
]{charter} 




% El títulos de la memoria, se usa en la carátula y se puede usar el cualquier lugar del documento con el comando \ttitle
\titulo{Sistema de Monitoreo de Temperaturas de Bogies - Locomotora CNR - CKD 8} 

% Nombre del posgrado, se usa en la carátula y se puede usar el cualquier lugar del documento con el comando \degreename
\posgrado{Carrera de Especialización en Sistemas Embebidos} 
%\posgrado{Carrera de Especialización en Internet de las Cosas} 
%\posgrado{Carrera de Especialización en Intelegencia Artificial}
%\posgrado{Maestría en Sistemas Embebidos} 
%\posgrado{Maestría en Internet de las cosas}

% Tu nombre, se puede usar el cualquier lugar del documento con el comando \authorname
\autor{Ing. Hernán Gomez Molino} 

% El nombre del director y co-director, se puede usar el cualquier lugar del documento con el comando \supname y \cosupname y \pertesupname y \pertecosupname
\director{Esp. Ing. Martín Harris}
\pertenenciaDirector{SOFSE - DNT} 
% FIXME:NO IMPLEMENTADO EL CODIRECTOR ni su pertenencia
\codirector{Ing. Guillermo Figini} % para que aparezca en la portada se debe descomentar la opción codirector en el documentclass
\pertenenciaCoDirector{SOFSE - DNT}

% Nombre del cliente, quien va a aprobar los resultados del proyecto, se puede usar con el comando \clientename y \empclientename
\cliente{Esp. Ing. Mariano Fernández Soler}
\empresaCliente{CENADIF}

% Nombre y pertenencia de los jurados, se pueden usar el cualquier lugar del documento con el comando \jurunoname, \jurdosname y \jurtresname y \perteunoname, \pertedosname y \pertetresname.
\juradoUno{Nombre y Apellido (1)}
\pertenenciaJurUno{pertenencia (1)} 
\juradoDos{Nombre y Apellido (2)}
\pertenenciaJurDos{pertenencia (2)}
\juradoTres{Nombre y Apellido (3)}
\pertenenciaJurTres{pertenencia (3)}
 
\fechaINICIO{24 de junio de 2021}		%Fecha de inicio de la cursada de GdP \fechaInicioName
\fechaFINALPlan{19 de agosto de 2021} 	%Fecha de final de cursada de GdP
\fechaFINALTrabajo{20 de junio de 2022}	%Fecha de defensa pública del trabajo final


\begin{document}

\maketitle
\thispagestyle{empty}
\pagebreak


\thispagestyle{empty}
{\setlength{\parskip}{0pt}
\tableofcontents{}
}
\pagebreak


\section*{Registros de cambios}
\label{sec:registro}


\begin{table}[ht]
\label{tab:registro}
\centering
\begin{tabularx}{\linewidth}{@{}|c|X|c|@{}}
\hline
\rowcolor[HTML]{C0C0C0} 
Revisión & \multicolumn{1}{c|}{\cellcolor[HTML]{C0C0C0}Detalles de los cambios realizados} & Fecha      \\ \hline
0      & Creación del documento                                 &\fechaInicioName \\ \hline
%1      & Se completa hasta el punto 4 inclusive                 & dd/mm/aaaa \\ \hline
%2      & Se completa hasta el punto 7 inclusive
%		  Se puede agregar algo más \newline
%		  En distintas líneas \newline
%		  Así                                                    & dd/mm/aaaa \\ \hline
%3      & Se completa hasta el punto 11 inclusive                & dd/mm/aaaa \\ \hline
%4      & Se completa el plan	                                 & dd/mm/aaaa \\ \hline
\end{tabularx}
\end{table}

\pagebreak



\section*{Acta de constitución del proyecto}
\label{sec:acta}

\begin{flushright}
Buenos Aires, \fechaInicioName
\end{flushright}

\vspace{2cm}

Por medio de la presente se acuerda con el \authorname\hspace{1px} que su Trabajo Final de la \degreename\hspace{1px} se titulará ``\ttitle'', consistirá esencialmente en la implementación del prototipo de un sistema que gestionará la información y alarmas de las temperaturas relevantes de los pares montados de la locomotora, y tendrá un presupuesto preliminar estimado de \textcolor{red}{600} hs de trabajo y \textcolor{red}{\$XXX}, con fecha de inicio \fechaInicioName\hspace{1px} y fecha de presentación pública \fechaFinalName.

Se adjunta a esta acta la planificación inicial.

\vfill

% Esta parte se construye sola con la información que hayan cargado en el preámbulo del documento y no debe modificarla
\begin{table}[ht]
\centering
\begin{tabular}{ccc}
\begin{tabular}[c]{@{}c@{}}Dr. Ing. Ariel Lutenberg \\ Director posgrado FIUBA\end{tabular} & \hspace{2cm} & \begin{tabular}[c]{@{}c@{}}\clientename \\ \empclientename \end{tabular} \vspace{2.5cm} \\ 
\multicolumn{3}{c}{\begin{tabular}[c]{@{}c@{}} \supname \\ Director del Trabajo Final\end{tabular}} \vspace{2.5cm} \\
%\begin{tabular}[c]{@{}c@{}}\jurunoname \\ Jurado del Trabajo Final\end{tabular}     &  & \begin{tabular}[c]{@{}c@{}}\jurdosname\\ Jurado del Trabajo Final\end{tabular}  \vspace{2.5cm}  \\
%\multicolumn{3}{c}{\begin{tabular}[c]{@{}c@{}} \jurtresname\\ Jurado del Trabajo Final\end{tabular}} \vspace{.5cm}                                                                     
\end{tabular}
\end{table}




\section{1. Descripción técnica-conceptual del proyecto a realizar}
\label{sec:descripcion}

%Objetivo del sistema original.
Las locomotoras CNR - CKD8 , incorporadas a la flota de Trenes Argentinos, tienen integrado en su equipamiento original un sistema denominado Hot Box.
El objetivo principal del sistema es evitar la operación de la locomotora en una condición de sobretemperatura de rodamientos en los bogies que pueda desencadenar en una falla catastrófica.
Para ello detecta la condición temprana de degradación mediante el monitoreo constante de la temperatura de operación en los puntos pasibles de falla.

%Descripción de sistema original.
El sistema está compuesto por:
\begin{itemize}
\item Dos cajas Hot Box ubicadas una en cada cabina de conducción, sobre el pupitre de mando. 
\item Un bus de sensores, protocolo 1Wire.
\item 36 sensores de temperatura, ubicados 6 en cada par montado.
\item Una conexión de comunicación entre ambas Hot Box.
\item Un conector frontal, ubicado en la Hot Box de la cabina 1, que permite la conexión de un sensor.
\end{itemize}
La disposición de estos componentes se muestra esquemáticamente en la figura 1.
\begin{figure}[h]
	\centering
	\includegraphics[angle=0,width=0.95\textwidth]{./Figuras/SMTB_bloques.pdf}
	\caption{Diagrama en bloques del sistema}
	\label{fig:mesh1}
\end{figure}

Si bien las cajas presentan funcionalidades distintas, ambas integran un microcontrolador, 5 teclas idénticas y presentan la información al conductor mediante un display de 7 segmentos, de 6 dígitos, y una matriz de LEDs de 6 x 6.
Sus diferencias consisten en la conexión al bus y el conector frontal, que solo están implementados en la Hot Box 1.

%Problemas
El sistema original adolece de diversos problemas a la hora de realizar su mantenimiento, a saber:
\begin{itemize}
\item Diseño cerrado. La información puesta a disposición por el fabricante para la reparación de los componentes es inexistente en los manuales provistos. Los repuestos disponibles para adquisición en la fábrica solo contemplan el recambio de unidades completas.
\item Costos de reposición. Los repuestos deben importarse del país de origen, lo que impacta fuertemente en los costos de adquisición. El fabricante es, además, el único proveedor del sistema a nivel global.
\item Soporte de postventa. El fabricante no provee la posibilidad de realizar cambios o mejoras al sistema. 
\end{itemize}

%Descripción del proyecto.
Con la motivación de corregir estos problemas, se propone desarrollar un reemplazo del sistema de diseño propio, con la máxima integración de componentes nacionales posible y reutilizando la instalación existente en las locomotoras. Adicionalmente, este desarrollo permitirá la nacionalización de las cadenas de valor que siempre resulta ventajosa para el país. La designación del producto será Sistema de Monitoreo de Temperaturas de Bogies (SMTB).

Especificamente, el desarrollo comprenderá el reemplazo de ambas unidades Hot Box por las que denominaremos Unidad de Control e Interfaz (UCI) y de los sensores de temperatura.
Las UCI conservarán el factor de forma original y su alimentación eléctrica de manera que el reemplazo de sistema solo implique el recambio de las unidades, sin tener que intervenir la instalación existente. El hardware será rediseñado para utilizar la tecnología nacional seleccionada para el proyecto.

El software embebido de las UCI (SUCI) contará con las funcionalidades actuales, más el agregado de algunas que mejorarán sus posibilidades de uso. Las funciones principales serán:
\begin{itemize}
\item El sistema deberá gestionar la información de los 36 sensores instalados en los bogies.
	\item Registrará de manera no volátil (EEPROM) la correspondencia del ID de sensor con la ubicación en que se encuentra.
	\item Presentará la información de temperatura y ubicación de la lectura al conductor de la locomotora.
	\item Manejará las alarmas que se requieran para cada ubicación.
	\item Permitirá la operación degradada (sensor/es o UCI fuera de servicio).
	\item Proveerá soporte a las tareas de mantenimiento del sistema:
	\begin{enumerate}
	\item Lazo de autochequeo del sistema. Es la comprobación de la existencia y consistencia de registros de sensores en ambas UCI, y comunicación con los sensores registrados.
	\item Alta de sensor nuevo (registro de ID y ubicación). 
	\item Alta de UCI\#1 nueva (carga de registro EEPROM actual utilizando UCI\#2 como soporte).
	\item Alta de UCI\#2 nueva. (carga de registro EEPROM actual utilizando UCI\#1 como soporte).
	\end{enumerate}
	\item Proveerá soporte a las tareas de mantenimiento mecánico de material rodante: 
	\begin{enumerate}
	\item Cambio de uno o ambos bogies. Esto requerirá un procedimiento de alta de sensores sin necesidad de desinstalarlos.
	\item Instalación de UCI utilizada en otra locomotora. Esto requerirá un procedimiento para resolver el conflicto de configuración a nivel EEPROMs.   
	\end{enumerate}
\end{itemize}
 
\section{2. Identificación y análisis de los interesados}
\label{sec:interesados}

%
\begin{table}[ht]
%\caption{Identificación de los interesados}
%\label{tab:interesados}
\begin{tabularx}{\linewidth}{@{}|l|X|X|l|@{}}
\hline
\rowcolor[HTML]{C0C0C0} 
Rol           & Nombre y Apellido & Organización 	& Puesto 	\\ \hline
%Auspiciante   &                   &              	&        	\\ \hline
Cliente       & \clientename      &\empclientename	&   Gte de Innovación\\ \hline
%Impulsor      &                   &              	&        	\\ \hline
Responsable   & \authorname       & FIUBA        	& Alumno 	\\ \hline
Colaboradores & Ing. Guillermo Figini&SOFSE - DNT  	& Codirector Trabajo Final\\ \hline
Orientador    & \supname	      & \pertesupname 	& Director Trabajo Final \\ \hline
Equipo        & Ing. Germán Zaupa \newline 
				Esteban Landeira          &SOFSE - DNT        	&         	\\ \hline
%Opositores    &                   &              	&        	\\ \hline
Usuario final & Conductores y mecánicos de locomotoras&       	&        	\\ \hline
\end{tabularx}
\end{table}

\begin{itemize}
	\item Equipo: Ing.Germán Zaupa, estará a cargo del diseño mecánico de las UCI y los sensores.
	\item Equipo: Esteban Landeira, coordinará las pruebas sobre locomotoras.
\end{itemize}

\section{3. Propósito del proyecto}
\label{sec:proposito}

El propósito de este proyecto es desarrollar un prototipo del sistema descripto que permita validar el diseño, ser homologado y producido localmente para su uso en las locomotoras de Trenes Argentinos. Adicionalmente, se pretende mejorar el soporte y la mantenibilidad del sistema durante su ciclo de vida.


\section{4. Alcance del proyecto}
\label{sec:alcance}

El proyecto incluye las siguientes actividades:
\begin{itemize}
\item Diseño del hardware (UCI\_1 y UCI\_2)(*1).
\item Desarrollo del software(UCI\_1 y UCI\_2)(*1 y *2).
\item Diseño de los componentes mecánicos (cajas UCI y encapsulado de sensores).
\item Construcción de un prototipo.
\item Pruebas de ajuste y aceptación.
\end{itemize}
No se incluyen las siguientes tareas: 
\begin{itemize}
\item El desarrollo de un sistema de control de temperatura de rodamientos.
\item La producción en serie del sistema.
\end{itemize}
(*1) Se prevé la posibilidad de utilizar un solo hardware para ambas UCI, lo que permitiría reducir el stock de repuestos necesarios. En ese caso, un solo software debería cubrir ambas posibilidades con la incorporación de algún medio de selección del rol de la UCI. Por ejemplo, utilizar un jumper.

(*2) Se prevé la posibilidad de utilizar un software que simule el funcionamiento de los periféricos para pruebas del sistema.

\section{5. Supuestos del proyecto}
\label{sec:supuestos}

Para el desarrollo del presente proyecto se supone que:

\begin{itemize}
	\item Se va a contar con disponibilidad de dos unidades EDU-CIAA-NXP o CIAA-NXP en su defecto, como así también los componentes necesarios para el avance del proyecto.
	\item Se va a contar con autorización y disponibilidad para realizar pruebas en locomotoras.
\end{itemize}

%Por ejemplo, se podrían incluir supuestos respecto a disponibilidad de tiempo y recursos humanos y materiales, sobre la factibilidad técnica de distintos aspectos del proyecto, sobre otras cuestiones que sean necesarias para el éxito del proyecto como condiciones macroeconómicas o reglamentarias.
%\end{consigna}

\section{6. Requerimientos}
\label{sec:requerimientos}

\begin{consigna}{red}
Los requerimientos deben numerarse y de ser posible estar agruparlos por afinidad, por ejemplo:

\begin{enumerate}
	\item Requerimientos funcionales
		\begin{enumerate}
			\item El sistema debe...
			\item Tal componente debe...
			\item El usuario debe poder...
		\end{enumerate}
	\item Requerimientos de documentación
		\begin{enumerate}
			\item Requerimiento 1
			\item Requerimiento 2 (prioridad menor)
		\end{enumerate}
	\item Requerimiento de testing...
	\item Requerimientos de la interfaz...
	\item Requerimientos interoperabilidad...
	\item etc...
\end{enumerate}

Leyendo los requerimientos se debe poder interpretar cómo será el proyecto y su funcionalidad.

Indicar claramente cuál es la prioridad entre los distintos requerimientos y si hay requerimientos opcionales. 

No olvidarse de que los requerimientos incluyen a las regulaciones y normas vigentes!!!

Y al escribirlos seguir las siguientes reglas:
\begin{itemize}
	\item Ser breve y conciso (nadie lee cosas largas). 
	\item Ser específico: no dejar lugar a confusiones.
	\item Expresar los requerimientos en términos que sean cuantificables y medibles.
\end{itemize}

\end{consigna}

\section{7. Historias de usuarios (\textit{Product backlog})}
\label{sec:backlog}

\begin{consigna}{red}
Descripción: En esta sección se deben incluir las historias de usuarios y su ponderación (\textit{history points}). Recordar que las historias de usuarios son descripciones cortas y simples de una característica contada desde la perspectiva de la persona que desea la nueva capacidad, generalmente un usuario o cliente del sistema. La ponderación es un número entero que representa el tamaño de la historia comparada con otras historias de similar tipo.

El formato propuesto es: "como [rol] quiero [tal cosa] para [tal otra cosa]."

Se debe indicar explícitamente el criterio para calcular los \textit{story points} de cada historia
\end{consigna}

\section{8. Entregables principales del proyecto}
\label{sec:entregables}

\begin{consigna}{red}

Los entregables del proyecto son (ejemplo):

\begin{itemize}
	\item Manual de uso
	\item Diagrama de circuitos esquemáticos
	\item Código fuente del firmware
	\item Diagrama de instalación
	\item Informe final
	\item etc...
\end{itemize}

\end{consigna}

\section{9. Desglose del trabajo en tareas}
\label{sec:wbs}

\begin{consigna}{red}
El WBS debe tener relación directa o indirecta con los requerimientos.  Son todas las actividades que se harán en el proyecto para dar cumplimiento a los requerimientos. Se recomienda mostrar el WBS mediante una lista indexada:

\begin{enumerate}
\item Grupo de tareas 1
	\begin{enumerate}
	\item Tarea 1 (tantas hs)
	\item Tarea 2 (tantas hs)
	\item Tarea 3 (tantas hs)
	\end{enumerate}
\item Grupo de tareas 2
	\begin{enumerate}
	\item Tarea 1 (tantas hs)
	\item Tarea 2 (tantas hs)
	\item Tarea 3 (tantas hs)
	\end{enumerate}
\item Grupo de tareas 3
	\begin{enumerate}
	\item Tarea 1 (tantas hs)
	\item Tarea 2 (tantas hs)
	\item Tarea 3 (tantas hs)
	\item Tarea 4 (tantas hs)
	\item Tarea 5 (tantas hs)
	\end{enumerate}
\end{enumerate}

Cantidad total de horas: (tantas hs)

Se recomienda que no haya ninguna tarea que lleve más de 40 hs. 

\end{consigna}

\section{10. Diagrama de Activity On Node}
\label{sec:AoN}

\begin{consigna}{red}
Armar el AoN a partir del WBS definido en la etapa anterior. 

%La figura \ref{fig:AoN} fue elaborada con el paquete latex tikz y pueden consultar la siguiente referencia \textit{online}:

%\url{https://www.overleaf.com/learn/latex/LaTeX_Graphics_using_TikZ:_A_Tutorial_for_Beginners_(Part_3)\%E2\%80\%94Creating_Flowcharts}

\end{consigna}

\begin{figure}[htpb]
\centering 
\includegraphics[width=.8\textwidth]{./Figuras/AoN.png}
\caption{Diagrama en \textit{Activity on Node}}
\label{fig:AoN}
\end{figure}

Indicar claramente en qué unidades están expresados los tiempos.
De ser necesario indicar los caminos semicríticos y analizar sus tiempos mediante un cuadro.
Es recomendable usar colores y un cuadro indicativo describiendo qué representa cada color, como se muestra en el siguiente ejemplo:



\section{11. Diagrama de Gantt}
\label{sec:gantt}

\begin{consigna}{red}

Existen muchos programas y recursos \textit{online} para hacer diagramas de gantt, entre los cuales destacamos:

\begin{itemize}
\item Planner
\item GanttProject
\item Trello + \textit{plugins}. En el siguiente link hay un tutorial oficial: \\ \url{https://blog.trello.com/es/diagrama-de-gantt-de-un-proyecto}
\item Creately, herramienta online colaborativa. \\\url{https://creately.com/diagram/example/ieb3p3ml/LaTeX}
\item Se puede hacer en latex con el paquete \textit{pgfgantt}\\ \url{http://ctan.dcc.uchile.cl/graphics/pgf/contrib/pgfgantt/pgfgantt.pdf}
\end{itemize}

Pegar acá una captura de pantalla del diagrama de Gantt, cuidando que la letra sea suficientemente grande como para ser legible. 
Si el diagrama queda demasiado ancho, se puede pegar primero la ``tabla'' del Gantt y luego pegar la parte del diagrama de barras del diagrama de Gantt.

Configurar el software para que en la parte de la tabla muestre los códigos del EDT (WBS).\\
Configurar el software para que al lado de cada barra muestre el nombre de cada tarea.\\
Revisar que la fecha de finalización coincida con lo indicado en el Acta Constitutiva.

En la figura \ref{fig:gantt}, se muestra un ejemplo de diagrama de gantt realizado con el paquete de \textit{pgfgantt}. En la plantilla pueden ver el código que lo genera y usarlo de base para construir el propio.

\begin{figure}[htbp]
\begin{center}
\begin{ganttchart}{1}{12}
  \gantttitle{2020}{12} \\
  \gantttitlelist{1,...,12}{1} \\
  \ganttgroup{Group 1}{1}{7} \\
  \ganttbar{Task 1}{1}{2} \\
  \ganttlinkedbar{Task 2}{3}{7} \ganttnewline
  \ganttmilestone{Milestone o hito}{7} \ganttnewline
  \ganttbar{Final Task}{8}{12}
  \ganttlink{elem2}{elem3}
  \ganttlink{elem3}{elem4}
\end{ganttchart}
\end{center}
\caption{Diagrama de gantt de ejemplo}
\label{fig:gantt}
\end{figure}


\begin{landscape}
\begin{figure}[htpb]
\centering 
\includegraphics[height=.85\textheight]{./Figuras/Gantt-2.png}
\caption{Ejemplo de diagrama de Gantt rotado}
\label{fig:diagGantt}
\end{figure}

\end{landscape}

\end{consigna}


\section{12. Presupuesto detallado del proyecto}
\label{sec:presupuesto}

\begin{consigna}{red}
Si el proyecto es complejo entonces separarlo en partes:
\begin{itemize}
	\item Un total global, indicando el subtotal acumulado por cada una de las áreas.
	\item El desglose detallado del subtotal de cada una de las áreas.
\end{itemize}

IMPORTANTE: No olvidarse de considerar los COSTOS INDIRECTOS.

\end{consigna}

\begin{table}[htpb]
\centering
\begin{tabularx}{\linewidth}{@{}|X|c|r|r|@{}}
\hline
\rowcolor[HTML]{C0C0C0} 
\multicolumn{4}{|c|}{\cellcolor[HTML]{C0C0C0}COSTOS DIRECTOS} \\ \hline
\rowcolor[HTML]{C0C0C0} 
Descripción &
  \multicolumn{1}{c|}{\cellcolor[HTML]{C0C0C0}Cantidad} &
  \multicolumn{1}{c|}{\cellcolor[HTML]{C0C0C0}Valor unitario} &
  \multicolumn{1}{c|}{\cellcolor[HTML]{C0C0C0}Valor total} \\ \hline
 &
  \multicolumn{1}{c|}{} &
  \multicolumn{1}{c|}{} &
  \multicolumn{1}{c|}{} \\ \hline
 &
  \multicolumn{1}{c|}{} &
  \multicolumn{1}{c|}{} &
  \multicolumn{1}{c|}{} \\ \hline
\multicolumn{1}{|l|}{} &
   &
   &
   \\ \hline
\multicolumn{1}{|l|}{} &
   &
   &
   \\ \hline
\multicolumn{3}{|c|}{SUBTOTAL} &
  \multicolumn{1}{c|}{} \\ \hline
\rowcolor[HTML]{C0C0C0} 
\multicolumn{4}{|c|}{\cellcolor[HTML]{C0C0C0}COSTOS INDIRECTOS} \\ \hline
\rowcolor[HTML]{C0C0C0} 
Descripción &
  \multicolumn{1}{c|}{\cellcolor[HTML]{C0C0C0}Cantidad} &
  \multicolumn{1}{c|}{\cellcolor[HTML]{C0C0C0}Valor unitario} &
  \multicolumn{1}{c|}{\cellcolor[HTML]{C0C0C0}Valor total} \\ \hline
\multicolumn{1}{|l|}{} &
   &
   &
   \\ \hline
\multicolumn{1}{|l|}{} &
   &
   &
   \\ \hline
\multicolumn{1}{|l|}{} &
   &
   &
   \\ \hline
\multicolumn{3}{|c|}{SUBTOTAL} &
  \multicolumn{1}{c|}{} \\ \hline
\rowcolor[HTML]{C0C0C0}
\multicolumn{3}{|c|}{TOTAL} &
   \\ \hline
\end{tabularx}%
\end{table}


\section{13. Gestión de riesgos}
\label{sec:riesgos}

\begin{consigna}{red}
a) Identificación de los riesgos (al menos cinco) y estimación de sus consecuencias:
 
Riesgo 1: detallar el riesgo (riesgo es algo que si ocurre altera los planes previstos de forma negativa)
\begin{itemize}
	\item Severidad (S): mientras más severo, más alto es el número (usar números del 1 al 10).\\
	Justificar el motivo por el cual se asigna determinado número de severidad (S).
	\item Probabilidad de ocurrencia (O): mientras más probable, más alto es el número (usar del 1 al 10).\\
	Justificar el motivo por el cual se asigna determinado número de (O). 
\end{itemize}   

Riesgo 2:
\begin{itemize}
	\item Severidad (S): 
	\item Ocurrencia (O):
\end{itemize}

Riesgo 3:
\begin{itemize}
	\item Severidad (S): 
	\item Ocurrencia (O):
\end{itemize}


b) Tabla de gestión de riesgos:      (El RPN se calcula como RPN=SxO)

\begin{table}[htpb]
\centering
\begin{tabularx}{\linewidth}{@{}|X|c|c|c|c|c|c|@{}}
\hline
\rowcolor[HTML]{C0C0C0} 
Riesgo & S & O & RPN & S* & O* & RPN* \\ \hline
       &   &   &     &    &    &      \\ \hline
       &   &   &     &    &    &      \\ \hline
       &   &   &     &    &    &      \\ \hline
       &   &   &     &    &    &      \\ \hline
       &   &   &     &    &    &      \\ \hline
\end{tabularx}%
\end{table}

Criterio adoptado: 
Se tomarán medidas de mitigación en los riesgos cuyos números de RPN sean mayores a...

Nota: los valores marcados con (*) en la tabla corresponden luego de haber aplicado la mitigación.

c) Plan de mitigación de los riesgos que originalmente excedían el RPN máximo establecido:
 
Riesgo 1: plan de mitigación (si por el RPN fuera necesario elaborar un plan de mitigación).
  Nueva asignación de S y O, con su respectiva justificación:
  - Severidad (S): mientras más severo, más alto es el número (usar números del 1 al 10).
          Justificar el motivo por el cual se asigna determinado número de severidad (S).
  - Probabilidad de ocurrencia (O): mientras más probable, más alto es el número (usar del 1 al 10).
          Justificar el motivo por el cual se asigna determinado número de (O).

Riesgo 2: plan de mitigación (si por el RPN fuera necesario elaborar un plan de mitigación).
 
Riesgo 3: plan de mitigación (si por el RPN fuera necesario elaborar un plan de mitigación).

\end{consigna}


\section{14. Gestión de la calidad}
\label{sec:calidad}

\begin{consigna}{red}
Para cada uno de los requerimientos del proyecto indique:
\begin{itemize} 
\item Req \#1: copiar acá el requerimiento.

\begin{itemize}
	\item Verificación para confirmar si se cumplió con lo requerido antes de mostrar el sistema al cliente. Detallar 
	\item Validación con el cliente para confirmar que está de acuerdo en que se cumplió con lo requerido. Detallar  
\end{itemize}

\end{itemize}

Tener en cuenta que en este contexto se pueden mencionar simulaciones, cálculos, revisión de hojas de datos, consulta con expertos, mediciones, etc.  Las acciones de verificación suelen considerar al entregable como ``caja blanca'', es decir se conoce en profundidad su funcionamiento interno.  En cambio, las acciones de validación suelen considerar al entregable como ``caja negra'', es decir, que no se conocen los detalles de su funcionamiento interno.

\end{consigna}

\section{15. Procesos de cierre}    
\label{sec:cierre}

\begin{consigna}{red}
Establecer las pautas de trabajo para realizar una reunión final de evaluación del proyecto, tal que contemple las siguientes actividades:

\begin{itemize}
	\item Pautas de trabajo que se seguirán para analizar si se respetó el Plan de Proyecto original:
	 - Indicar quién se ocupará de hacer esto y cuál será el procedimiento a aplicar. 
	\item Identificación de las técnicas y procedimientos útiles e inútiles que se emplearon, y los problemas que surgieron y cómo se solucionaron:
	 - Indicar quién se ocupará de hacer esto y cuál será el procedimiento para dejar registro.
	\item Indicar quién organizará el acto de agradecimiento a todos los interesados, y en especial al equipo de trabajo y colaboradores:
	  - Indicar esto y quién financiará los gastos correspondientes.
\end{itemize}

\end{consigna}


\end{document}
